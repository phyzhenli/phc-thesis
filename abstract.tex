\begin{abstract}
    人工智能的快速发展需要海量的算力支持,导致计算需求急剧增加,消耗了大量能源。同时,在可穿戴设备、便携设备和数据中心等场景中,集成电路的功耗问题日益严峻,亟需新的设计方法用以降低系统功耗,提高芯片能效。
    % ,研究者需要寻找新的方法以降低系统功耗,提高芯片能效。
    近似计算作为一种新兴的计算范式,允许系统在可接受的误差范围内完成计算任务。与容错应用结合,它能够在满足精度需求的前提下提高计算效率,降低芯片能耗。
    % 因此,在数字信号处理、机器学习等场景中,近似计算得到了工业界和学术界的广泛关注。
    
    作为近似计算的一个分支,近似电路设计旨在通过对电路中的精确算术单元引入近似,以降低硬件开销。
    % 乘法是一种常见的计算操作,在许多应用中被频繁调用。为了提高计算效率,研究人员提出了近似乘法器,对乘法操作在硬件上进行优化。
    其中,作为最常见的基本计算单元,乘法器的近似优化是研究的重点。
    然而,
    % 现有的近似乘法器相关工作一方面没有同时考虑数据分布和输入极性,另一方面基于低效的手工设计方法。
    现有的近似乘法器相关设计不仅缺乏对数据统计学特性的利用,同时存在手工设计效率低以及在不同架构上硬件收益不匹配的缺点。
    因此本文提出了两种开源的自动化近似乘法器设计方法,能够高效地生成不同精度的适用于专用集成电路(Application-Specific Integrated Circuit, ASIC)和现场可编程门阵列(Field Programmable Gate Array, FPGA)的高质量近似乘法器。基于得到的近似乘法器,本文利用强化学习方法对深度神经网络(Deep Neural Network, DNN)硬件加速器进行了近似逻辑综合研究。具体内容如下:
    
    (1)面向 ASIC,本文基于数据分布和输入极性提出了一种开源的自动化近似乘法器设计方法,该方法能够高效地生成适应于特定应用的高性能ASIC近似乘法器,提高应用的运算效率。
    % 具体来说,
    本文提出的方法在对乘法器的部分积累加求和前,引入与、或、异或和移位操作来压缩部分积,降低部分积阵列的规模,减轻累加压力。
    % 本文利用改进的Baugh-Wooley算法支持了补码有符号乘法器。
    为了能够利用计算机自动化求解,本文将寻找较优压缩操作的问题建模成数学问题,
    % 该问题的目标函数同时考虑了乘法器的精度和硬件开销,
    并利用混合整数遗传算法对电路结构进行搜索。
    % 本文对所提出的方法进行了实验,
    实验结果表明,针对均匀分布和三个不同规模的DNN生成的8比特无符号乘法器领先于国际前沿工作。
    % 针对基于8比特无符号数量化的三个不同规模的DNN生成的乘法器,在精度损失不超过0.01\%的情况下实现了26.4\%-47.6\%的硬件性能收益。
    针对自适应滤波器生成的16位补码有符号乘法器在峰值信噪比(Peak Signal-to-Noise Ratio, PSNR)损失较小的情况下实现了27.1\%的硬件成本提升。针对32比特半正态分布的实验结果表明提出的方法对大位宽乘法器同样有效。
    
    (2)由于ASIC和FPGA底层架构不同,ASIC近似乘法器通常无法在FPGA上取得相同比例的硬件性能提升。因此面向FPGA应用,本文基于贝叶斯优化提出了一种开源的自动化近似乘法器生成方法,避免了手工修改查找表编码方法效率较低的问题。
    该方法假设乘法器的部分积在生成后、累加前存在一次由半加器阵列进行的压缩操作。
    % (与(1)中的压缩操作不同,这里是半加操作)。
    本文利用贝叶斯优化,基于所提出的4种半加器简化方法对半加器阵列进行优化。优化后,该方法保留后续累加过程中部分积的粗粒度加法,使其能够被电子设计自动化(Electronic Design Automation, EDA)工具高效地识别并映射到FPGA的快速进位链。与国际前沿工作中的1167个近似乘法器相比,本文生成的乘法器位于帕累拖前沿,精度和硬件开销综合指标平均提高了28.70\%-38.47\%。
    
    (3)结合前面两个工作,面向大规模电路,本文基于强化学习方法和得到的近似乘法器进行了近似逻辑综合研究。
    % 具体来讲,
    本文首先提出了一个开源的基于最大无扇出锥(Maximum Fanout-Free Cone, MFFC)自适应超图划分的端到端强化学习逻辑优化框架,对大规模电路进行全面优化,以改善芯片的面积、延迟和功耗。
    % 该框架首先利用硬件描述语言解析工具Yosys对电路进行读入和解析,接着提取电路中组合逻辑,利用“自然划分”和MFFC超图划分将提取的组合逻辑分割成多个子电路,并利用强化学习序列优化方法对所有的子电路并行探索,最后由商业综合工具评估面积和延迟结果。
    本文对超过150个电路进行实验,结果表明所提出的方法与ABC resyn2 相比,面积延迟积平均提高了5.17\%。之后,本文将提出的强化学习逻辑优化框架与基于数据分布和输入极性得到的DNN近似乘法器库结合,对不同近似乘法器实现的DNN硬件加速器进行了实验,结果显示近似乘法器的单独硬件开销与对应加速器的硬件开销的改善比例存在一定偏差。然而,基于帕累拖前沿的乘法器实现的加速器仍处于帕累拖前沿。

    综上所述,本文针对近似乘法器,开展了设计方法和电路中的应用两方面研究,结果表明,面向大规模电路,硬件设计师在使用本文提出的两种自动化方法得到高能效近似乘法器后,应对得到的帕累拖前沿乘法器进行探索以确定最佳的硬件实现。

\end{abstract}
    
\begin{abstract*}
    The rapid development of artificial intelligence requires massive computational power support, leading to a sharp increase in computational demand and consuming a significant amount of energy. Meanwhile, in scenarios such as wearable devices, portable devices, and data centers, the power consumption issue of integrated circuits is becoming increasingly severe, necessitating new design methods to reduce system power consumption and improve chip efficiency. Approximate computing, as an emerging computing paradigm, allows systems to complete computing tasks within an acceptable range of error. When combined with fault tolerance applications, it can improve computing efficiency and reduce chip power consumption while meeting accuracy requirements.
    
    As a branch of approximate computing, approximate circuit design aims to introduce approximations to precise arithmetic units in circuits to reduce hardware costs. Among these, the approximate optimization of multipliers is a key focus of research as they are the most common basic computing units. However, existing designs of approximate multipliers not only lack utilization of data statistical characteristics but also suffer from low efficiency in manual design and mismatched hardware benefits across different architectures. Therefore, this thesis proposes two automated methods for designing approximate multipliers, which can efficiently generate high-quality approximate multipliers of different precisions suitable for Application-Specific Integrated Circuits (ASICs) and Field Programmable Gate Arrays (FPGAs). Building upon the obtained approximate multipliers, this thesis conducts research on approximate logic synthesis of Deep Neural Network (DNN) hardware accelerators using reinforcement learning methods. The specific details are as follows:

    (1) This thesis presents an open-source automated approximate multiplier design method tailored for ASICs based on data distribution and input polarity, which efficiently generates high-performance ASIC approximate multipliers tailored for specific applications, enhancing computational efficiency. The method proposed in this thesis introduces AND, OR, XOR, and shift operations to compress partial products before the partial accumulation summation, reducing the size of the partial product array and alleviating the accumulation pressure. To enable automated computer-aided design, the problem of finding optimal compression operations is formulated as a mathematical problem in this thesis, and a hybrid integer genetic algorithm is employed to search the circuit structure. Experimental results demonstrate that the 8-bit unsigned multipliers generated for uniform distributions and three different scales of DNNs outperform state-of-the-art approaches. For the 16-bit two's complement signed multipliers generated for adaptive filters, a 27.1\% hardware cost improvement is achieved with minimal Peak Signal-to-Noise Ratio (PSNR) loss. Results for 32-bit half-normal distributions indicate the effectiveness of the proposed method for large-bit-width multipliers as well.

    (2) Due to the differing underlying architectures of ASICs and FPGAs, ASIC approximate multipliers often cannot achieve the same level of hardware performance improvement on FPGAs. Therefore, targeting FPGA applications, this thesis proposes an open-source automated approximate multiplier generation method based on Bayesian optimization to address the inefficiency of manual modification of lookup table encoding methods. This method assumes a compression operation by a carry-save adder array on the partial products of the multiplier after generation and before accumulation. By utilizing Bayesian optimization and optimizing the carry-save adder array based on the four simplification methods proposed, this thesis retains coarse-grained addition of the partial products during subsequent accumulation, enabling efficient identification and mapping to the fast carry chains of FPGAs by Electronic Design Automation (EDA) tools. Compared to 1167 approximate multipliers in state-of-the-art works, the multipliers generated in this thesis are positioned at the Pareto frontier, with an average improvement of 28.70\%-38.47\% in the combined metrics of accuracy and hardware cost.

    (3) Incorporating the previous two works, this thesis conducts research on approximate logic synthesis for large-scale circuits using reinforcement learning methods based on obtained approximate multipliers. Initially, an open-source end-to-end reinforcement learning logic optimization framework based on adaptive hypergraph partitioning using Maximum Fanout-Free Cone (MFFC) is proposed to comprehensively optimize large-scale circuits, aiming to enhance chip area, delay, and power consumption. Experimental analyses on over 150 circuits demonstrate that the proposed method achieves an average area-delay product improvement of 5.17\% compared to ABC resyn2. Subsequently, the developed reinforcement learning logic optimization framework is combined with a DNN approximate multiplier library derived from data distribution and input polarity, and experiments are conducted on DNN hardware accelerators implemented with different approximate multipliers. Results indicate a certain deviation in the improvement ratio between the standalone hardware cost of approximate multipliers and the hardware cost of the corresponding accelerators. Nevertheless, accelerators implemented based on Pareto-optimal frontiers of multipliers still reside at the Pareto frontier.

    In conclusion, this thesis delves into both the design methods and circuit applications of approximate multipliers. Findings suggest that for large-scale circuits, hardware designers, after obtaining highly efficient approximate multipliers using the two automated methods proposed in this thesis, should explore the Pareto-optimal frontier multipliers to determine the optimal hardware implementation.
    
\end{abstract*}