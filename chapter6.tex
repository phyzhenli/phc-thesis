\chapter{总结与展望}

\section{总结}

作为一种新兴的计算范式,近似计算允许系统在可接受的误差范围内返回结果,与容错应用结合,能够提高计算效率,降低芯片能耗。针对近似乘法器,本文提出并开源了三个研究工作:
\begin{itemize}
    \item 面向ASIC,基于数据分布和输入极性提出并开源了一种自动化近似乘法器设计方法,该方法利用逻辑操作和移位操作在部分积生成后、累加前对部分积进行一次压缩,降低部分积阵列的规模,减轻后续的累加压力。对提出的方法进行了广泛的测试评估,结果显示,基于均匀分布下8比特无符号数生成的近似乘法器与国际前沿工作相比取得显著进步,在平均误差距离MAE和硬件开销PDA两个指标均领先;同时,对提出的方法基于不同规模的DNN以及滤波器应用进行实验,结果显示与国际前沿工作相比,生成的近似乘法器在几乎没有引起应用级精度下降的情况下PDA改善了26.4\%-27.1\%;基于32比特无符号数正态分布的实验结果证明了方法对大位宽乘法器的有效性。
    \item 面向FPGA,提出并开源了一个基于贝叶斯优化的自动化近似乘法器生成方法,该方法假设乘法器的部分积在生成后、累加前存在一次由半加器阵列进行的压缩操作,针对精确半加器提出了4种简化方案,利用贝叶斯算法对半加器阵列进行优化,保留压缩后累加过程中部分积的粗粒度加法。通过详细设计地能够同时反映误差和硬件开销的目标函数,与国际前沿工作中的 1167 个乘法器相比,生成的乘法器位于帕累拖前沿,在硬件成本和误差的乘积上平均有28.70\%-38.47\%的改进。
    \item 提出并开源了一个基于MFFC自适应超图划分的端到端强化学习逻辑优化框架,首先利用Yosys对电路进行读入和解析,接着将电路中的组合逻辑提取出来,利用“自然划分”和MFFC超图划分将组合逻辑分割成多个子电路,对所有的子电路基于ABC并行地用提出的强化学习序列优化方法在给定的时间内进行探索,并由商业综合工具进行评估。基于超过150个电路的实验结果显示,提出的方法处于国际前沿水平,与ABC resyn2 相比,面积延迟积ADP平均提高了5.17\%。进一步与提出的近似乘法器库结合,对DNN硬件加速器的近似实现进行了研究,结果显示近似乘法器的单独硬件成本与对应加速器的硬件成本存在一定偏差,因此在实际使用中可对库中的不同乘法器进行多次尝试,以实现更好的PPA。
\end{itemize}

\section{展望}

本文提出的两种近似乘法器设计方法均基于部分积的压缩实现,如何结合部分积的生成过程进行优化需要做进一步地讨论。同时,对生成的近似乘法器添加可以旁路的误差补偿模块能够满足更多的应用场景,可调精度近似乘法器的自动化生成方法值得探索。最后,在近似逻辑综合的研究中,序列探索时考虑输入的到达时间能够对网表进行更好的优化,可对逻辑综合工具添加相应的功能进行支持。