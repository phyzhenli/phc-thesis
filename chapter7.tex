\chapter{总结与展望}

\section{总结}

作为一种新兴的计算范式,近似计算允许系统在可接受的误差范围内返回结果,与容错应用结合,能够提高计算效率,降低芯片能耗。
针对近似乘法器的设计及应用,本文首先基于数据分布和输入极性提出了一个面向ASIC的开源的自动化设计方法,该方法能够高效地生成适合于给定应用的高能效近似乘法器,提高应用的计算效率。然而,由于底层架构的区别,ASIC近似乘法器通常无法在FPGA上获得相同程度的硬件性能提升。因此本文面向FPGA提出了一种开源的基于贝叶斯优化的自动化近似乘法器设计方法,能够在短时间内生成不同精度的高质量软核乘法器,提高FPGA应用的能效,并降低资源占用率。最后,结合得到的近似乘法器,本文基于强化学习利用近似逻辑综合技术研究了不同近似乘法器对大规模电路整体PPA的影响。具体内容如下:

\begin{itemize}
    \item 面向ASIC,本文基于数据分布和输入极性提出了一种开源的自动化近似乘法器设计方法,能够高效地生成适用于不同应用的高能效近似乘法器。
    % 该方法利用逻辑操作和移位操作在部分积生成后、累加前对部分积进行一次压缩,降低部分积阵列的规模,减轻后续的累加压力。对提出的方法进行了广泛的测试评估,
    基于均匀分布下8比特无符号数的实验结果显示,生成的近似乘法器与国际前沿工作相比取得显著进步,在平均误差距离MAE和硬件开销PDA两个指标均领先。基于不同规模的DNN以及滤波器应用的实验结果显示,生成的近似乘法器在几乎没有引起应用级精度下降的情况下PDA改善了26.4\%-27.1\%。基于32比特无符号数正态分布的实验结果证明了方法对大位宽乘法器的有效性。

    \item 面向FPGA,本文提出了一个开源的基于贝叶斯优化的自动化近似乘法器生成方法。
    % 该方法假设乘法器的部分积在生成后、累加前存在一次由半加器阵列进行的压缩操作,针对精确半加器提出了4种简化方案,利用贝叶斯算法对半加器阵列进行优化,保留压缩后累加过程中部分积的粗粒度加法。通过详细设计地能够同时反映误差和硬件开销的目标函数,
    与国际前沿工作中的 1167 个乘法器相比,生成的乘法器位于帕累托前沿,在硬件成本和误差的乘积上平均有28.70\%-38.47\%的改进。

    \item 结合前两个工作,本文利用强化学习进行了近似逻辑综合研究。首先提出了一个开源的基于MFFC自适应超图划分的端到端强化学习逻辑优化框架,实验结果显示,
    % 该框架采用Yosys对电路进行读入和解析,接着将电路中的组合逻辑提取出来,利用“自然划分”和MFFC超图划分将组合逻辑分割成多个子电路,对所有的子电路基于ABC并行地用提出的强化学习序列优化方法在给定的时间内进行探索,并由商业综合工具进行评估。基于超过150个电路的实验结果显示,提出的方法处于国际前沿水平,
    面积延迟积ADP比ABC resyn2平均提高了5.17\%。之后,本文将提出的面向大规模电路的强化学习逻辑优化框架与基于数据分布和输入极性生成的DNN近似乘法器库结合,对DNN硬件加速器的近似实现进行了研究,结果显示近似乘法器的单独硬件成本与对应加速器的硬件成本存在一定偏差。
    % ,因此在实际使用中可对库中的不同乘法器进行多次尝试,以实现更好的PPA。
\end{itemize}

\section{展望}

本文提出的两种近似乘法器设计方法均基于部分积的压缩实现,如何结合部分积的生成过程进行优化需要做进一步地讨论。同时,对生成的近似乘法器添加可以旁路的误差补偿模块能够满足更多的应用场景,可调精度近似乘法器的自动化生成方法值得探索。最后,在近似逻辑综合的研究中,序列探索时考虑输入的到达时间能够对网表进行更好的优化,可对逻辑综合工具添加相应的功能进行支持。